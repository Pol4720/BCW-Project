\documentclass[a4paper, 12pt]{article}
\usepackage[left=2.5cm, right=2.5cm, top=3cm, bottom=3cm]{geometry}
\usepackage[spanish]{babel}
\usepackage{amsmath}
\usepackage{graphicx}
\usepackage{color}
\usepackage{xcolor}
\usepackage[utf8]{inputenc}
\usepackage[T1]{fontenc}
\usepackage{listings}
\usepackage{tikz}
\usetikzlibrary{shapes,arrows,positioning}

\definecolor{colorgreen}{rgb}{0,0.6,0}
\definecolor{colorgray}{rgb}{0.5,0.5,0.5}
\definecolor{colorpurple}{rgb}{0.58,0,0.82}
\definecolor{colorback}{RGB}{255,255,204}
\definecolor{colorbackground}{RGB}{200,200,221}
%Definiendo el estilo de las porciones de codigo
\lstset{
 backgroundcolor=\color{colorbackground},
commentstyle=\color{colorgreen},
keywordstyle=\color{colorpurple},
numberstyle=\tiny\color{colorgray},
stringstyle=\color{colorgreen},
basicstyle=\ttfamily\footnotesize,
breakatwhitespace=false,
breaklines=true,
captionpos=b,
keepspaces=true,
numbers=left,
showspaces=false,
showstringspaces=false,
showtabs=false,
tabsize=2,
frame=single,
framesep=2pt,
rulecolor=\color{black},
framerule=1pt
}



\begin{document}
\graphicspath{{./}}

\begin{center}
\text{\huge \textbf{Informe del Proyecto Final} }\\
\vspace {0.5cm}
\text{\huge \textbf{de}}\\
\vspace {0.5cm}
\text{\huge \textbf{Estadística}}\\
\vspace {5cm}
\text{\huge Richard Alejandro Matos Arderí}\\
\text{\huge Mauricio Sunde Jiménez}\\
\vspace {1cm}
\vspace {2cm}
\text{\Large Grupo 311, Ciencia de la Computación.}\\
\vspace {0.5cm}
\text{\Large Facultad de Matemática y Computación}\\
\text{\Large Universidad de La Habana.}\\
\vspace {0.5cm}
\begin{figure}[h]
    \centering
    \includegraphics[width=0.2\textwidth, height=0.2\textheight]{MATCOM.jpg}
\end{figure}
\vspace {0.5cm}
\text{2024}\\


\end{center}

\newpage
\tableofcontents
\newpage

\section{Introducción}

\section{Análisis Descriptivo de los datos}

\begin{table}[ht]
    \centering
    \caption{Descripción de Variables del Dataset Breast Cancer Wisconsin (Diagnostics)}
    \label{tab:breast_cancer_variables}
    \resizebox{\textwidth}{!}{
        \begin{tabular}{|p{3.5cm}|p{5.5cm}|p{3.5cm}|p{3.5cm}|}
            \hline
            \textbf{Variable} & \textbf{Descripción} & \textbf{Clasificación Estadística} & \textbf{Escala de Medición} \\
            \hline
            ID & Identificador único del paciente & Categórica & Nominal \\
            diagnosis & Diagnóstico del tumor (M = maligno, B = benigno) & Categórica & Nominal \\
            radius\_mean & Radio medio del tumor en mm & Continua & Razón \\
            texture\_mean & Textura media del tumor & Continua & Razón \\
            perimeter\_mean & Perímetro medio del tumor en mm & Continua & Razón \\
            area\_mean & Área media del tumor en mm² & Continua & Razón \\
            smoothness\_mean & Suavidad media del tumor & Continua & Razón \\
            compactness\_mean & Compacidad media del tumor & Continua & Razón \\
            concavity\_mean & Concavidad media del tumor & Continua & Razón \\
            concave points\_mean & Puntos cóncavos medios del tumor & Continua & Razón \\
            symmetry\_mean & Simetría media del tumor & Continua & Razón \\
            fractal dimension\_mean & Dimensión fractal media del tumor & Continua & Razón \\
            radius\_se & Desviación estándar del radio del tumor & Continua & Razón \\
            texture\_se & Desviación estándar de la textura del tumor & Continua & Razón \\
            perimeter\_se & Desviación estándar del perímetro del tumor & Continua & Razón \\
            area\_se & Desviación estándar del área del tumor & Continua & Razón \\
            smoothness\_se & Desviación estándar de la suavidad del tumor & Continua & Razón \\
            compactness\_se & Desviación estándar de la compacidad del tumor & Continua & Razón \\
            concavity\_se & Desviación estándar de la concavidad del tumor & Continua & Razón \\
            concave points\_se & Desviación estándar de los puntos cóncavos del tumor & Continua & Razón \\
            symmetry\_se & Desviación estándar de la simetría del tumor & Continua & Razón \\
            fractal dimension\_se & Desviación estándar de la dimensión fractal del tumor & Continua & Razón \\
            radius\_worst & Radio máximo del tumor en mm & Continua & Razón \\
            texture\_worst & Textura máxima del tumor & Continua & Razón \\
            perimeter\_worst & Perímetro máximo del tumor en mm & Continua & Razón \\
            area\_worst & Área máxima del tumor en mm² & Continua & Razón \\
            smoothness\_worst & Suavidad máxima del tumor & Continua & Razón \\
            compactness\_worst & Compacidad máxima del tumor & Continua & Razón \\
            concavity\_worst & Concavidad máxima del tumor & Continua & Razón \\
            concave points\_worst & Puntos cóncavos máximos del tumor & Continua & Razón \\
            symmetry\_worst & Simetría máxima del tumor & Continua & Razón \\
            fractal dimension\_worst & Dimensión fractal máxima del tumor & Continua & Razón \\
            \hline
        \end{tabular}
    }
\end{table}


\newpage

\section{Análisis de la distribución}


\subsection{Estimación de parámetros}

\subsubsection{Estimación Puntual}

\subsubsection{Estimación por Intervalos}

\subsection{Pruebas de Hipótesis}


\section{Pruebas de Normalidad}

\subsubsection{Pruebas de Hipótesis para una población}

\subsubsection{Pruebas de Hipótesis para dos poblaciones}

\newpage

\section{Correlación e Independencia}


\end{document}