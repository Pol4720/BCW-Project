\documentclass[a4paper, 12pt]{article}
\usepackage[left=2.5cm, right=2.5cm, top=3cm, bottom=3cm]{geometry}
\usepackage[spanish]{babel}
\usepackage{amsmath}
\usepackage{graphicx}
\usepackage{color}
\usepackage{xcolor}
\usepackage[utf8]{inputenc}
\usepackage[T1]{fontenc}
\usepackage{listings}
\usepackage{tikz}
\usetikzlibrary{shapes,arrows,positioning}




\definecolor{colorgreen}{rgb}{0,0.6,0}
\definecolor{colorgray}{rgb}{0.5,0.5,0.5}
\definecolor{colorpurple}{rgb}{0.58,0,0.82}
\definecolor{colorback}{RGB}{255,255,204}
\definecolor{colorbackground}{RGB}{200,200,221}
%Definiendo el estilo de las porciones de codigo
\lstset{
 backgroundcolor=\color{colorbackground},
commentstyle=\color{colorgreen},
keywordstyle=\color{colorpurple},
numberstyle=\tiny\color{colorgray},
stringstyle=\color{colorgreen},
basicstyle=\ttfamily\footnotesize,
breakatwhitespace=false,
breaklines=true,
captionpos=b,
keepspaces=true,
numbers=left,
showspaces=false,
showstringspaces=false,
showtabs=false,
tabsize=2,
frame=single,
framesep=2pt,
rulecolor=\color{black},
framerule=1pt
}



\begin{document}
\graphicspath{{./}}

\begin{center}
\text{\huge \textbf{Informe del Proyecto Final} }\\
\vspace {0.5cm}
\text{\huge \textbf{de}}\\
\vspace {0.5cm}
\text{\huge \textbf{Estadística}}\\
\vspace {5cm}
\text{\huge Richard Alejandro Matos Arderí}\\
\text{\huge Mauricio Sunde Jiménez}\\
\vspace {1cm}
\vspace {2cm}
\text{\Large Grupo 311, Ciencia de la Computación.}\\
\vspace {0.5cm}
\text{\Large Facultad de Matemática y Computación}\\
\text{\Large Universidad de La Habana.}\\
\vspace {0.5cm}
\begin{figure}[h]
    \centering
    \includegraphics[width=0.2\textwidth, height=0.2\textheight]{MATCOM.jpg}
\end{figure}
\vspace {0.5cm}
\text{2024}\\


\end{center}

\newpage
\tableofcontents
\newpage

\section{Introducción}

\section{Análisis Descriptivo de los datos}


\newpage

\section{Análisis de la distribución}


\subsection{Estimación de parámetros}

\subsubsection{Estimación Puntual}

\subsubsection{Estimación por Intervalos}

\subsection{Pruebas de Hipótesis}

\subsubsection{Pruebas de Hipótesis para una población}

\subsubsection{Pruebas de Hipótesis para dos poblaciones}

\newpage

\section{Correlación e Independencia}


\end{document}












































